% test-gantt.tex
\documentclass[a4paper,landscape]{article}
\usepackage{pgfgantt}
\usepackage[landscape,margin=1cm]{geometry}

\begin{document}

\begin{ganttchart}[
                    hgrid,
                    vgrid,
                    time slot format=isodate,
                    expand chart=\textwidth,  % Ajusta automáticamente al ancho del texto
                    x unit=0.35cm, % Reducimos el tamaño de la unidad para Ajustar al ancho
                    % calendar week text={Semana~\currentweek} %Esto permite poner nombre "Semana" en vez de "Week"
                ]{2025-01-03}{2025-02-20}
    % Tres niveles de título: año-mes, semanas y días
    %\gantttitle{Title}{1} \\ % Título de la gráfica por mejorar
    \gantttitlecalendar{year, month=name, day, week, weekday} \\
    % WBS 1
    \ganttgroup{WBS 1: Planificación}{2025-01-03}{2025-01-24} \\
    \ganttbar{Tarea 1.1}{2025-01-03}{2025-01-10} \\
    \ganttbar{Tarea 1.2}{2025-01-11}{2025-01-17} \\
    \ganttbar{Tarea 1.3}{2025-01-18}{2025-01-24} \\
    \ganttmilestone{Hito 1}{2025-01-24} \\[2ex]
    % WBS 2
    \ganttgroup{WBS 2: Ejecución}{2025-01-25}{2025-02-20} \\
    \ganttbar{Tarea 2.1}{2025-01-25}{2025-02-05} \\
    \ganttbar{Tarea 2.2}{2025-02-06}{2025-02-13} \\
    \ganttbar{Tarea 2.3}{2025-02-14}{2025-02-20} \\
    \ganttmilestone{Hito Final}{2025-02-20}
\end{ganttchart}

\end{document}